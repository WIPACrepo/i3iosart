%!TEX TS-program = pdflatex
%!TEX root = i3det-top.tex
%!TEX encoding = UTF-8 Unicode

% (KDH) put overall electrical connection schematic diagram in 
% this section and include some brief details on cable specs 
% 19 AWG (to be confirmed) and 
 
\section{The Cable Systems}

The IceCube detector may be viewed as a digital network of optical sensors.  The
IceCube cable systems form the physical backbone of this network, supplying
simultaneously power to the DOMs and bi-directional digital communications between
the DOMs and the readout hardware at the surface.  Copper was chosen over optical 
fiber at an early stage in the project due to concerns with mechanical robustness
of fiber during freeze-in in the deep ice and a previous successful demonstration 
of the digitally-readout optical modules contained within the AMANDA detector 
(String 18).

The IceCube cable systems are broken down into the surface cables which stretch
from the ICL to surface junction boxes (SJBs) located near the top of each hole and
surface-to-DOM cables continuing from the SJBs into the deep ice.  All surface
cables and 85 of the 86 surface-to-DOM cables were manufactured by Ericsson, the 
Swedish telecommunications company.  The remaining surface-to-DOM cable, String 29,
was manufactured by JDR.  A depiction of the logical connection architecture
of the IceCube sensor network is shown in Figure~\ref{fig:icecube-cables-logical}.

\begin{figure}
\fbox{\vspace{2in}}
\label{fig:icecube-cables-logical}
\caption{Schematic of the IceCube cable network architecture.}
\end{figure}

 
\subsection{Surface Cables}
\subsection{Surface-to-DOM Cables}
\subsection{Surface Junction Boxes}
 