%!TEX TS-program = pdflatex
%!TEX root = i3det-top.tex
%!TEX encoding = UTF-8 Unicode

\section{Drilling and Deployment}

\subsection{Geometry Calibration}

The geometry of the detector is determined using drill and survey data
during deployment (stage 1), and then corrected and refined using the LED
flashers in ice (stage 2).
\subsubsection{Stage 1 Geometry Calibration}
The (x,y) coordinates of the string are calculated using the position
of the drill tower. Before deployment, when the drill tower is in position, at least three
of the tower corners are surveyed from at least one control point.
The coordinates for the center of the hole in the tower floor are
calculated from the corner coordinates. If x-y drift vs. depth has been calculated for the hole from drill
data, the drifts are added to the (x,y) coordinates of individual
DOMs, assuming that the string follows the hole center. 

The depth of the lowest DOM on the string is calculated using pressure
readings from the Paro/Keller pressure sensor, converted to depth by correcting
for the compressibility of the water in the hole and the ambient (air)
pressure measured before the pressure sensor hits the water. The
distance from the tower floor to the water surface is measured with a
laser ranger. The vertical DOM spacings are also measured with a laser
ranger. 

\subsubsection{Stage 2 Geometry Calibration}


The LED flashers are used to correct the relative depths of the
strings. This correction is typically less than 1~m relative to the
stage 1 data, but can be as large as 20~m (larger than the 17~m DOM
spacing) in cases where the pressure sensor fails during string
deployment before it takes the final depth reading. The correction is
calculated by finding the leading edge of the time distribution of the
light recorded by the
receiving DOM, denoted $t_0$. The distance corresponding to the
leading edge time is $d = c_{ice} \cdot t_0$, and the distances for
all receiving DOMs are plotted as a function of the vertical distance
between the flasher and the receiver, $z' = z_{receiver} -
z_{flasher}$. The resulting plot described a hyperbola, $d = \sqrt{D^2
+ (z' -\Delta z)^2}$, where $D$ is the horizontal distance between the
flasher string and the receiver string, calculated from stage~1 data,
and $\Delta z$ is the relative offset between the depths of the
flashing and receiving string. The hyperbola fit is done
simultaneously for each flashing string and all surrounding receiving
strings in order to calculate the relative offsets, which are then
applied to the z coordinate of all DOMs on the string.

{\it Trilateration}

Flasher corrections to the (x,y) coordinates of some DOMs in the
center of the DeepCore subarray were calculated using the
trilateration method. In this analysis, the 5~DOMs closest to the
flasher on each of the three closest strings surrounding the flasher are selected,
and a circle of radius $r = \sqrt{(d)^2 - (\Delta z)^2)}$ is drawn
around each receiving DOM, where $d$ is the distance between the DOM and the flasher calculated from the leading edge time of the received
light, and  $\Delta z$ is the relative depth of the flashing and
receiving DOMs calculated from the method described above. The
intersection points of all the circles are calculated, and the (x,y)
position of the flashing DOM is taken to be the average of the
centroid of the intersection points. The error bars on the positions
are $1 \sigma$ from a Gaussian fit to all centroid values; the x and y
corrdinates are fitted independently. The shifts relative to the
deployment data are found to be less than 1~m, and agree with the
drill head coordinates within the error bars.

