%!TEX TS-program = pdflatex
%!TEX root = i3det-top.tex
%!TEX encoding = UTF-8 Unicode

\section{IceCube Online Systems}
\textsl{(John K; 12-15 pages)}

%The division between \textit{triggering} and
%\textit{filtering} is the urgency of the processing.
%Triggers must operate within a bounded time and if not
%data is lost.  Filters operate behind such large buffers
%that this is not a consideration.  Historically this means
%that the triggers have been rather \emph{dumb} but there
%is in principle no ceiling to the trigger complexity.

\subsection{Data Flow Overview}

\subsection{SPS and SPTS}

\subsection{Data Readout and Timing}
\subsubsection{Communications and Cable Bandwidth}
\subsubsection{Master Clock System}
\subsubsection{DOR Card and Linux Driver}

\subsection{Processing at the Surface}
\textsl{(Dave G; 2-3 pages)}
\subsubsection{DOMHub and Hit Spooling}
\subsubsection{Supernova System}
\subsubsection{Triggers}
\subsubsection{Event Building}
\subsubsection{The Secondary Streams (SN/Moni/TCAL)}
\subsubsection{Configuration}
\subsubsection{Distributed Network Control}

\subsection{Online Filtering}
\textsl{(Erik B.; 3-4 pages)}
\subsubsection{Overview}
Big picture view of what PnF is and does (route data from DAQ output to
files for pickup by data handling, applying calibration, recos and
filters along the way, TFT content, etc)

\subsubsection{System Design}
Cover the overall system design and tech used (CORBA, I3Inlet/I3Outlets,
etc) and how all pieces fit together and are controlled (mention
pf2live). Words about historical operation in Plan B in early seasons,
plan A.

\subsubsection{Components}
A brief description of each PnF component
\paragraph{I3DAQDispatch}
\paragraph{PnF Server}
\paragraph{PnF Clients}
\paragraph{DB Cache}
\paragraph{Realtime Followup server and clients}

\subsubsection{Performance}
\paragraph{Overall rates and known limits (server + I/O limits)}
\paragraph{Average system latency}


\subsection{Data Handling}