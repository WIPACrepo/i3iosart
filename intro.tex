%!TEX TS-program = pdflatex
%!TEX root = i3det-top.tex
%!TEX encoding = UTF-8 Unicode

\section{Introduction}
\label{sec:intro}

\subsection{IceCube Science}
Construction of an observatory to examine the celestial phenomena using
neutrinos instead of photons: this objective was likely first conceived not long
after the mere existence of neutrinos was theorized by Pauli. In the six decades
following the experimental verification of the neutrino by Cowan and Reines,
detectors have been realized in many forms that have steadily progressed toward
the goal of multimessenger neutrino astronomy. From early first detections of
atmospheric neutrinos in deep underground telescopes \cite{Witwatersrand,KGF} to
radiochemical solar neutrino detectors characterizing the flux of neutrinos from
nuclear fusion in the Sun\cite{Homestake, GALLEX}, eventually leading to the
identification of neutrino mass, to the detection of a handful of neutrino
events from SN1987A\cite{SK1987A,IMB1987A,BUST1987A}, to the advanced
atmospheric and solar neutrino observatories of the 1990's that gave definitive
evidence of neutrino mass and constrained neutrino mixing
parameters\cite{SK,SNO}, the study of neutrinos emanating from astrophysical
processes has not failed to inform, and sometimes surprise, the fields of
astrophysics and particle physics.

Notwithstanding the events seen from SN1987A, a remarkable but regrettably
extremely rare class of phenomena, the detection of neutrinos originating in
astrophysical processes outside our solar system requires detector facilities of
extreme dimensions to detect the faint fluxes of these weakly-interacting
particles. The motivations for such massive observatories are numerous, however.
Neutrinos, because of their weak neutral character are useful probes of high
energy phenomena in the Universe. Unlike photons, their origins in astrophysical
acceleration sites unambiguously point to, literally, hadronic acceleration and
provide identification of the origins of cosmic rays. They arrive upon detection
undeflected and unscattered and thus point back to their originators and provide
a clear view of the physics deep within shrouded and compact sources. At extreme
energies of several PeV and above they are the only particles which can reach 
us from sources at cosmological distances.  The primary mission of IceCube has 
been discovery of astrophysical neutrinos, achieved in 2013\cite{HESE1}, and 
characterization of the sources.  At the rate of roughly a single detection
per year of neutrinos with energy above 1 PeV, it may take years to resolve the 
sources and the physics underlying the phenomena of high-energy cosmic 
radiations.  Whether or not IceCube achieves this goal and its many other
science objectives, detection of dark matter and other exotic particles,
study of neutrino oscillation physics, detection of the neutrino shockwave
from a galactic core collapse supernova, there are strong motivations for a 
future facility of larger scale to continue to bring exciting and surprising
science from multimessenger astronomy.


\subsection{A Functional Description of the IceCube Instrument}

In order to observe astrophysical neutrinos, %the primary science goal of the
experiment, IceCube exploits the fact that charged particles moving through the
ice at super-luminal speed emit Cherenkov photons. An enormous detection volume
is required since the cross-sections of neutrinos are small for producing
secondary charged particles in interactions with ordinary matter. The glacial
ice cap at the South Pole is about \SI{3}{\kilo\meter} thick and therefore
predestined as operation site since it is not only offering aplenty interaction
material but also a medium with unmatched high quality. Cherenkov light is
produced in cascades of neutrino-induced muons penetrating the deep, other
high-energy particles as well as generated by by cosmic-ray muons that result
from cosmic-ray interactions in the atmosphere above Antarctica. Due to a
Cherenkov photon yield of $\mathcal{O}(\num{E5})$ visible photons per
\SI{}{\giga\electronvolt} of shower energy, the long optical attenuation length
in South Pole ice and large-area photomultipliers (PMTs) it is possible to
instrument cubic kilometers of ice with a rather wide spacing of detectors. The
basic detection unit in IceCube in order to capture the Cherenkov light is the
digital optical module or \textit{DOM} which is covered in great detail in
Sec.~\ref{sec:dom}. Encapsulated in a \SI{1/2}{''} thick glass pressure sphere
to withstand the extreme pressure in the deep ice, the main components of a DOM
are a \SI{10}{''} PMT, embedded high-voltage generation, a flasher calibration
board, and a mainboard containing the analog and digital processing circuitry
for PMT pulses. %The transmission of the digital data is controlled by an FPGA
and embedded processor hosted on the mainboard. The digitized data is fed to a
central computing facility at the surface via a unique cable system, see
Sec.~\ref{sec:cable}. Aspects of detector deployment and ice drilling are
covered in Sec.~\ref{sec:drill-deploy}. An overview of the data flow as well as
its readout, processing and filtering are subjects of Sec.~\ref{sec:online}
where we also cover the data handling, monitoring and operational performance of
the observatory. The IceCube instrument consists of three sub-detectors --
IceCube, DeepCore and IceTop -- using the same instrumentation design of
embedded digital optical modules and associated surface readout. A schematic
layout of the array is shown in Fig.~\ref{fig:array}

\begin{figure}[!h]
 \centering
 \includegraphics[width=0.8\textwidth]{graphics/intro/ArrayWSeasonsLabels_crop.pdf}
 \caption{The IceCube Neutrino Observatory with its sub-array DeepCore and the air shower array IceTop.}
 \label{fig:array}
\end{figure}


\subsubsection{IceCube}
In order to detect the Cherenkov photons emitted by charged particles traversing the ice, \num{5160} DOMs are deployed between \SI{1450}{\meter} and \SI{2450}
{\meter} below the glacial surface on \num{86} vertical strings each holding \num{60} DOMs deployed along a copper cable. The main \textit{deep-ice} 
array consists of \num{78} strings with a vertical separation of the DOMs on each string of \SI{17}{\meter}. The strings are deployed on a hexagonal grid with
\SI{125}{\meter} horizontal spacing and spanning a volume of one cubic kilometer of ice. 
This design was chosen in order to meet the primary science goal of detecting astrophysical neutrinos in the energy range of 
$\mathcal{O}(\SI{}{\tera\electronvolt}) - \mathcal{O}(\SI{}{PeV})$. The observation of these neutrinos was achieved by searching
for events that start inside the volume of IceCube and deposit more than \SI{30}{\tera\electronvolt} of energy \cite{IC3:evidence}. 
Two different event types form the standard signatures of neutrinos in IceCube.
Track-like events originate from a charged current interaction of a
high-energy muon-neutrino with a nucleus, producing a muon with
its typical Cherenkov cone along the track together with a hadronic cascade at the vertex.
An angular reconstruction of such a muon track and hence the incident neutrino direction
can reach a precision better than $ \SI{1}{\degree}$ which is confirmed by a pointing-resolution analysis
of the moon shadow \cite{IC3:moon}. Energy loss above the minimum-ionizing regime, typically $ \sim \SI{1}{\tera\electronvolt}$ in IceCube, is highly
dominated by stochastic processes resulting in large fluctuations in the amount of energy deposited for
different muons of the same energy. Another class of events are electro-magnetic or hadronic cascades from all
neutrino flavors, resulting in a spherical light deposition in the detector.
Since the total light output of such a cascade is directly proportional to its energy and the
cascades are generally well contained in the detector, the energy reconstruction for such
events is much more precise than for track-like events. The average deposited energy
resolution for both event types combined is $ \sim \SI{15}{\%}$ \cite{IC3:ereco}.


\subsubsection{DeepCore}

The remaining subset of in-ice DOMs is deployed in the deep ice below a depth of \SI{1750}{\meter} forming a denser instrumented volume. This sub-array, the DeepCore \cite{ICECUBE:DC} consists of eight specialized and closely spaced strings of sensors located around the central IceCube string. 
Its inter-string spacings are \SI{75}{\meter} and inter-DOM spacings are \SI{7}{\meter}.
Most of the PMTs in this region have \SI{35}{\%} higher quantum efficiency than the standard IceCube modules. This results in an energy threshold being about an order of magnitude below IceCube, of about \SI{10}{\giga\electronvolt}. This design is optimized for the detection of atmospheric neutrinos with energies typically in the range from \SI{10}{\giga\electronvolt} to \SI{400}{\tera\electronvolt} \cite{ICECUBE:AtmNu}.


\subsubsection{IceTop}

The air shower array IceTop \cite{ICECUBE:IceTop} consists of \num{81} Cherenkov tanks filled with clear ice that are arranged in pairs on the same, approximately \SI{125}{\meter}, triangular grid on which the in-ice array is deployed. The two tanks at each surface station are separated from each other by \SI{10}{\meter}. Each tank contains two standard IceCube DOMs. Air showers initiated in the atmosphere by cosmic rays are typically spread over a number of stations. The light generated in the tanks by the shower particles (electrons, photons, muons and hadrons) is a measure of the energy deposit of these particles in the tanks. IceTop is sensitive to primary cosmic rays of energies in the range of \SI{}{PeV} to \SI{}{EeV}  with a resolution of \SI{25}{\%} at \SI{2}{PeV}, which improves to \SI{12}{\%} above \SI{10}{PeV} \cite{IT:measurement}.




