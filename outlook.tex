\section{Outlook}

IceCube's online systems continually evolve in order to enhance the
capabilities and performance of the detector and meet new science needs.  
For example, the 2016 physics run start
included a new low-energy IceTop trigger, a new filter designed to search for
moderate-velocity magnetic monopoles, and a transition to a more flexible
realtime alert system that transmits candidate neutrino events at a rate of
3 mHz via I3MS to a followup analysis server in the Northern Hemisphere.
Specialized filter streams for rare events likely to be astrophysical in
origin have also been added within the past two years.  

In addition to online software improvements and regular computing upgrades, a
number of calibration and maintenance hardware activities are 
planned. For example, in order to mitigate the effects of continued snow
accumulation on the IceTop tanks --- increased energy threshold and reduced
sensitivity to the electromagnetic air shower component --- a small number
of scintillator-based  detector panels have been installed at existing
IceTop stations, and development of additional panels is underway.

The stability of the primary triggers and filters and a high detector
uptime enable continued detection of additional astrophysical neutrino
candidates.  Searches for neutrino point sources are ongoing, and a robust
realtime followup campaign facilitates multi-wavelength observations of
reconstructed neutrino source directions, with the eventual goal of
definitive identification of the astrophysical hadron accelerators.

\subsection{IceCube--Gen2 and PINGU}

As neutrino astronomy matures, designs for next-generation detectors are underway.   Neutrino
detectors under construction or planned for the Northern Hemisphere include
KM3NeT in the Mediterranean \cite{km3net} and GVD in Lake Baikal \cite{gvd}.  

IceCube--Gen2 is an experiment under design for the South Pole
consisting of a high-energy in-ice array, a surface air shower array with
additional veto capabilities, a low-energy in-ice infill array (the Precision IceCube Next
Generation Upgrade, or PINGU), and potentially a shallow sub-surface 
array of radio antennas \cite{gen2_whitepaper}.  The high-energy array features an 
increased string spacing that allows instrumentation of up to 10 times larger 
effective target volume relative to IceCube; the scientific focus will be the
detection and characterization of astrophysical neutrino sources at the PeV
energy scale.  The PINGU sub-array \cite{pingu_loi} will densely instrument
6 MTon of ice in the center of DeepCore, enabling precision neutrino
oscillation measurements down to the GeV energy range, determination of the
neutrino mass ordering, and dark matter searches at energies above a
few GeV.  Updated calibration devices will be deployed in the new
boreholes in order to measure the optical
properties of the ice more precisely and improve event reconstruction. An extended surface
array, potentially several times larger in diameter than the high-energy
array, will be used both for cosmic ray studies and to veto downgoing
atmospheric muons and neutrinos.  Radio-frequency detection of ultra-high-energy
neutrino-induced showers in the ice is a relatively recent
technique which shows considerable promise to achieve effective
target volumes of about 100 times IceCube at $\unit[10^{18}]{eV}$ where neutrinos
originating from scattering of ultra-high-energy cosmic rays on the 
cosmic microwave background are expected \cite{ara2}.  The 
combination of optical and radio-frequency technologies offers the possibility
of a broadband neutrino observatory with diverse astrophysics and particle
physics science capabilities. 

The baseline sensor design for the array is a modernized version of the
IceCube DOM~\cite{pingu_loi}.  Mechanical components such as the glass sphere and
penetrator, as well as the high-quantum-efficiency PMT, remain unchanged,
while the triggering and digitization electronics are being redesigned.
Alternative sensor designs are also under study that increase photocathode
area, photon collection, angular coverage, and/or directional resolution.
Slimmer, more cylindrical profiles may 
allow smaller hole diameters, decreasing drilling time and fuel usage.
The recent development of the capability to deliver cargo and fuel to the
station via overland traverse rather than aircraft will reduce fuel
costs. This 
is part of an overall effort to reduce logistical support requirements
compared to IceCube construction.  

The IceCube detector has achieved ``first light'' in neutrino astronomy and
has the capability to continue operating at least until the end of the next decade, supporting 
a diverse neutrino physics and astrophysics program and providing unique datasets
to the scientific community.  The IceCube--Gen2 facility will continue this legacy 
and contribute to further discoveries in neutrino astronomy and multi-messenger astrophysics.

