\section{Outlook}

IceCube's online systems continued to be improved in order to enhance the
capabilities of the detector.  For example, the 2016 physics run start
included a new low-energy IceTop trigger, a new filter designed to search for
moderate-velocity magnetic monopoles, and a transition to a more flexible
realtime alert system that transmits candidate neutrino events via I3MS to a
followup analysis server in the Northern Hemisphere.

EXAMPLE FUTURE?

In addition to software improvements and regular computing upgrades, a
number of calibration and maintenance hardware activities are 
planned. For example, in order to mitigate the effects of continued snow
accumulation on the IceTop tanks --- increased energy threshold and reduced
sensitivity to the electromagnetic air shower component --- a small number
of scintillator-based  detector panels have been installed at existing
IceTop stations, and development of additional panels is underway.

The stability of the primary triggers are filters and the high detector
uptime enables continued detection of additional astrophysical neutrino
candidates.  Searches for neutrino point sources are ongoing, and a robust
realtime followup campaign facilitates multi-wavelength observations of
reconstructed neutrino source directions, with the eventual goal of
definitive identification of the astrophysical accelerators.

\subsection{IceCube--Gen2 and PINGU}

As neutrino astronomy matures and moved beyond this original discovery phase, designs
for next-generation detectors are underway.  Neutrino detectors under construction
or planned for the Northern Hemisphere include KM3Net in the Mediterranean
\cite{km3net} and GVD in Lake Baikal \cite{gvd}.  

IceCube--Gen2 is a flagship experiment under design for the South Pole
\cite{gen2_whitepaper} consisting of a high-energy in-ice array, a surface air
shower array / veto, and a low-energy in-ice infill array (PINGU).  

CONTINUE MORE INFO

The baseline sensor design for the array is a modernized version of the
IceCube DOM.  Mechanical components such as the glass sphere and
penetrator, as well as the high-quantum-efficiency PMT, remain unchanged,
while the triggering and digitization electronics are being redesigned.
Alternative sensor designs are also under study that increase photocathode
area and/or angular coverage.  Slimmer, more cylindrical profiles may
allow smaller hole diameters, decreasing drilling time and fuel usage.  This
is part of an overall effort to reduce logistical support requirements
compared to IceCube construction.  



