\section{\label{sec:dom_calibration} DOM Calibration and Stability}

% FIX ME add intro

\subsection{\label{sec:domcal} DOMCal}

The calibration of the PMT waveforms recorded by the DOMs, i.e. translation
of digitized samples into voltage and time, as well as the gain of the PMT
itself, is achieved via DOM-by-DOM calibration constants determined by the
modules themselves.  The calibration software, DOMCal, uses precision calibration
circuits on the DOM Main Board as well as single photons as inputs
for a ladder of calibration routines.  Analysis and fitting of the
calibration results is done by the DOMCal software, with the results being
transmitted from each DOM to the surface as an XML file of fit parameters.

%The ladder of calibration routines is shown in
%Fig.~\ref{fig:domcal_ladder}.
The primary reference inputs to the calibration are a precision electronic
pulser circuit providing known charges; a circuit providing a reference DC
bias voltage level; the 20 MHz oscillator on the Main
Board, used as the timing reference; and single photoelectrons, either from
ambient ``dark noise'' photons or a low-luminosity LED on the Main Board.

Because the operating conditions for
the in-ice DOMs are so stable, DOMCal is only run once a year on the full
detector. IceTop DOMs are calibrated once per month.  Calibration is
typically performed on half the detector at at time, the other half
remaining in data-taking mode.  Because the calibration procedure produces
light, however, these runs are not used in normal analysis.

First, the discriminator circuits used to trigger the DOM are calibrated
using the electronic pulser circuit.  This calibration is later refined using actual
PMT waveforms, once the PMT gain is known.  Next, the ATWD voltage levels
are calibrated by sweeping the input DC bias voltage and recording the
response; because of the slight variations in the ATWD circuits, this
calibration is produced for every sample and every channel.

This ATWD calibration also in principle allows determination of the baseline
voltage levels of each channel, needed for charge integration.  However, in
practice, these baselines are extremely sensitive to the operation
condition of the DOMs, and since data-taking conditions cannot be exactly
replicated while running DOMCal, the baselines used during data-taking are
determined instead by using averaged forced triggers taken during a normal
run.  DOMCal can still use its own baselines, though, for charge integration
during the calibration procedure.

The highest-gain channels of the two ATWDs are calibrated using the electronic
pulser circuit, and then the gains of the other ATWD channels and the FADC
are determined by varying the pulser output and comparing the charges
simultaneously recorded in multiple gain channels.  This relative
calibration is later refined using PMT waveforms stimulated by the Main
Board LED.

% FIX ME put in gains

% FIX ME explain quadratic relationship

The ATWD sampling speed is calibrated by digitizing the Main Board oscillator
waveform and recording the number of clock cycles as a function of ATWD
speed setting.  The FADC sampling speed is slaved to the 20 MHz
oscillator, which is used as a timing reference.  The relative timing of
the ATWD and FADC waveforms is determined using the electronic pulser circuit;
non-linear fitting of the digitized waveforms to known templates is
required in order to determine the FADC offset to the required accuracy.
The transit time of the PMT and delay board as a function of PMT high
voltage is determined by calculating the delay between the digitized
current pulse through the Main Board LED and the observed light pulse in
the ATWDs.   

The PMT gain as a function of high voltage is calibrated using ambient
``dark noise'' photons --- the charge $e$ prior to amplification is quantized
and known.  At each voltage level, a histogram of many waveform charges is recorded,
and each histogram is fit with an exponential plus Gaussian model (see
Fig.~\ref{fig:domcal_hvfit}).  The peak of the Gaussian component is used to
determine the amplification of the PMT at each voltage, and a linear fit
of $\log_{10}(\mathrm{gain})$ versus $\log_{10}(\mathrm{voltage})$ allows
the high voltage of each PMT to be tuned to desired operating point ($10^7$
for in-ice DOMs; see Fig.~\ref{fig:domcal_hv_settings}).  Small ($3-5\%$)
corrections to the gain of each DOM are determined using charge
distributions recorded during normal data-taking; this corrects for a small
systematic difference in the charge as determined by DOMCal's integration
and the waveform pulse unfolding used in data processing.

% From JK DOMCal Hist.ipynb
\begin{figure}[!h]
 \centering
 \includegraphics[width=0.6\textwidth]{graphics/dom/domcal/hvfit.pdf}
 \caption{Sample SPE charge spectrum as recorded by DOMCal and fit
   \textit{in situ} with a Gaussian+exponential model.} 
 \label{fig:domcal_hvfit}
\end{figure}

% From JK DOMCal HV Settings.ipynb
\begin{figure}[!h]
 \centering
 \includegraphics[width=0.6\textwidth]{graphics/dom/domcal/inice_hv_2016.pdf}
 \caption{PMT high voltage at $10^7$ gain for in-ice DOMs.}
 \label{fig:domcal_hv_settings}
\end{figure}

\subsection{\label{sec:waveformcal} Waveform Calibration and Droop
  Correction}
An ATWD waveform consists of readouts from 128 independent digitizers,
while an FADC waveform consists of successive outputs of a 6-stage
pipelined digitizer (3 bins for SLC launches and 256 for HLC
launches). Two calibration constants are needed to turn each of these
raw digitizer readout values (an integer number of ADC counts) into a
voltage, from which the deposited charge is calculated. The deposited
charge is the basis for all energy measurements in IceCube. The
calibration constants needed are 1) the pedestal, the value the digitizer reads when the input voltage is zero, and
2) the gain, the input voltage required to increase increase the readout value by one count. 

Since each bin of each ATWD channel is read from an independent
digitizer, it is important to distinguish between the pedestals of
individual bins and the common pedestal of the entire waveform. The
baseline is the mean of the pedestals of each bin in a waveform. For
the FADC, this is the same as the pedestal. The pedestal pattern is
the deviation of each bin's pedestal from the common baseline. For the
FADC, this is identically zero. Since 2008, the DOM has subtracted the
pedestal pattern from
the ATWD waveform before sending it to the surface. The pedestal pattern is computed at the beginning of each
run by comparing 25 averaged pedestals. The autocorrelation coefficent
between the pairs of averaged pedestals is computed to detect light
contamination in the pedestals, and the shift between the baseline of
the pairs is calculated to determine that the baseline is stable. This
procedure ensures that fewer than 1 DOM in 1000 runs will contain a
contaminated baseline. Thereafter, both ATWD
and FADC waveforms can be calibrated by first subtracting the common
baseline from each bin, then multiplying by the gain. Correct
measurement of the common baseline is critical to correct charge
measurement and energy reconstruction.

The baseline is set to about 10\% of the maximum value of the
digitizer counts in
order to capture signals that go below the baseline. Since 2012, the baseline value is set by the DAQ configuration in order to ensure
stability. The baseline value differs for each digitizer channel in
each DOM, ranging from 112 to 161 counts in the fACD and 109 to 172
counts in the ATWD. The baselines for each digitizer channel in each DOM are measured with
beacon hits, forced triggers which are collected at a rate of 1.2~Hz
per DOM
in in-ice DOMs and 4.8~Hz in IceTop DOMs. Beacon waveforms
from the FADC and ATWD of a typical DOM are shown in Figure~\ref{fig:raw_baselines}.

\begin{figure}[!h]
  \captionsetup[subfigure]{labelformat=empty}
  \centering
  \subfloat[]{\includegraphics[width=0.5\textwidth]{graphics/dom/reliability/fadc_average_beacon_raw.pdf}}
  \subfloat[]{\includegraphics[width=0.5\textwidth]{graphics/dom/reliability/ATWD_average_beacon_raw.pdf}}
  \caption{Beacon waveforms from the FADC (left) and ATWD (right) of
    an IceCube DOM. The waveforms are an average of about 1000 beacon
   launches. The baseline, which is the mean value of the
    beacon waveform, is shown as a horizontal red line. Typical raw SPE
    waveforms are inset.}
  \label{fig:raw_baselines}
\end{figure}

The gain is measured by DOMCal (Sec.~\ref{sec:domcal}), a single value
for the FADC waveform and a bin-dependent value for the ATWD
waveform. The calibrated waveform voltage is then

\begin{equation}
  \mathrm{voltage} = \mathrm{(ADC\ counts - baseline)*gain}
\end{equation}

The waveform start time is then corrected for the PMT transit time,
the average time it takes a pulse to propagate through the entire
PMT. The PMT transit time correction $t_{transit}$ is dependent on the PMT high
voltage:

\begin{equation}
  t_{transit} = \frac{m}{\sqrt{HV}} + b
\end{equation}

where $m$ and $b$ are determined by DOMCal. The typical value of $m$
is 2000~ns$\cdot \sqrt{\mathrm{V}}$ and the typical value of $b$ is
80~ns, which includes the 75~ns delay line of the mainboard. The
typical transit time is therefore 130~ns. 

Waveform start times from the second ATWD chip and the FADC are further
corrected for the delay $\Delta t$ with respect to the first ATWD
chip, so the total start time correction is $t_{transit} + \Delta t$.

Finally, the waveforms are corrected for the effects of droop from the
transformer that couples the mainboard to the PMT output. The toroid
coupling effectively acts as a high-pass filter on the PMT output
which makes the tails of the waveforms ``droop'' and even
undershoot. This effect is temperature dependent and is worse at lower
temperatures. The droop correction inverts the effect of the high-pass
filter and eliminates the undershoot in the waveform tails. This is
done by calculating the expected reaction voltage from the toroid at
each time, and adding the reaction voltage to the calibrated waveform
to compensate. The reaction voltages are made to decay exponentially
according to a temperature-dependent model of the transformer’s
behavior. When a readout contains consecutive launches from the same
DOM, the reaction voltages at the end of the last launch are used to
correct for the residual droop in the follow-on launch. IceCube DOMs
use two types of toroid transformers: the ``old toroid'' with a short
time constant which was used in early DOM production, and a ``new
toroid'' with a longer time constant that produces less
distortion. The  full correction is modeled with a dual time constant
where the DOM's transient response $\tilde{\delta}(t)$ to an input
signal $\delta(t)$ is given by

\begin{equation}
\tilde{\delta}(t) = \delta (t) - N((1 - f) e^{t/\tau_1} +f
e^{t/\tau_2})
\end{equation}

The first time constant $\tau_1$ is given by 
\begin{equation}
\tau_1(T) = A + \frac{B}{1 + \exp{-T/C}}
\end{equation}
where $T$ is temperature and $A$, $B$ and $C$, $N$ and $f$ are determined empirically.

For old toroid DOMs, the second time constant $\tau_2 =
0.75\tau_1$. For new toroid DOMs, the second time constant is 500~ns. 

\subsection{\label{sect:dom:rapcal}RAPCal}

The Reciprocal Active Pulsing calibration (RAPCal) is a
continuously-running method for translating hit timestamps from the the
individual free-running DOM clock domains to the clock domain in the
IceCube Lab, which is synchronized to UTC.  A bipolar pulse is sent to each
DOM over the power/communications wire pair and then returned, with each
transmission and receive point timestamped locally.  The symmetry of the down-
and up-transmission allows a translation from each DOM clock domain to the
surface clock domain at the midpoint, without prior knowledge of cable lengths.
The base implementation is described in Ref.~\cite{ICECUBE:DAQ}; we
describe here the details of the time translation algorithm and validation
of the DOM relative synchronization. 

\begin{figure}[!h]
 \centering
 \includegraphics[width=0.4\textwidth]{graphics/dom/rapcal/rapcal_symmetry.pdf}
 \caption{The Reciprocal Active Pulsing calibration (RAPCal) allows
   translation from the free-running clocks of each DOM to the GPS-slaved
   clocks in the IceCube lab.  Each transmit and receive pair is
   timestamped in the local clock domain, and by the symmetry of the
   situation the midpoints are synchronous.}
 \label{fig:rapcal_symmetry}
\end{figure}

\begin{figure}[h]
 \centering
 \includegraphics[width=0.6\textwidth]{graphics/dom/rapcal/dom_wf_zero_crossing.pdf}
 \caption{Digitized RAPCal pulse as received by a DOM, after cable dispersion.  The
   zero-crossing of the baseline-subtracted pulse is used as a fine-delay
   correction to the received timestamp.}
 \label{fig:rapcal_zero_crossing}
\end{figure}

A RAPCal pulse sequence to each DOM results in a series of four timestamps
$T_{\mathrm{tx}}^{\mathrm{DOR}}$, $T_{\mathrm{rx}}^{\mathrm{DOM}}$, 
$T_{\mathrm{tx}}^{\mathrm{DOM}}$,  and $T_{\mathrm{rx}}^{\mathrm{DOR}}$
(Fig.~\ref{fig:rapcal_symmetry}), with timestamps in 20 MHz units
of the DOM and surface (DOR, see Sect.~\ref{sect:online:master_clock}) clocks.  The received dispersed bipolar pulses
are digitized by the communications ADCs and 
timestamped.  A ``fine delay'' correction $f$ to
$T_{\mathrm{rx}}^{\mathrm{DOM}}$ and $T_{\mathrm{rx}}^{\mathrm{DOR}}$ is calculated by interpolating
to find the negative-going zero crossing, relative to a baseline voltage
calculated using the initial samples of the waveform (Fig.~\ref{fig:rapcal_zero_crossing}):

\begin{equation}
  \tilde{T}_{\mathrm{rx}} = T_{\mathrm{rx}} - f\ .
\end{equation}


\noindent The midpoints $T_C^{\mathrm{DOR,DOM}}$ halfway between $T_\mathrm{tx}^{\mathrm{DOR,DOM}}$ and
$\tilde{T}_\mathrm{rx}^{\mathrm{DOR,DOM}}$ are identified as synchronous.

To translate an arbitrary DOM timestamp $t$ to UTC
time, we use the RAPCal results bracketing $t$ to derive a linear
relationship

\begin{equation}
  \mathrm{UTC}(t) = (1+\epsilon)(t - T_C^{\mathrm{DOM}}) +
  T_C^{\mathrm{DOR}} + \Delta\ .
\end{equation}

\noindent The slope $(1+\epsilon)$ accounts for drifts in the 20 MHz DOM
clocks and is calculated from the midpoints $T_C$ of the bracketing RAPCal
results: 

\begin{equation}
  1+\epsilon = \frac{T_{C,2}^{\mathrm{DOR}} -
    T_{C,1}^{\mathrm{DOR}}}{T_{C,2}^{\mathrm{DOM}} -
    T_{C,1}^{\mathrm{DOM}}}\ .
\end{equation}

\noindent Finally, because the timestamps $T^{\mathrm{DOR}}$ count the offset into
the current UTC second, the UTC time offset $\Delta$ of the previous
1-second boundary, provided by the master clock, is added.

The stability and repeatability of the calibration is monitored by
tracking the cable delay from multiple measurements, defined as

\begin{equation}
  T_{\mathrm{cable}} = \frac{1}{2} \left( ( T_{\mathrm{rx}}^{\mathrm{DOR}} -
  T_{\mathrm{tx}}^{\mathrm{DOR}} ) - (1+\epsilon)(T_{\mathrm{tx}}^{\mathrm{DOM}} -
  T_{\mathrm{rx}}^{\mathrm{DOM}} )\right) \ .
\end{equation}

\noindent A representative distribution of $T_{\mathrm{cable}}$ from one DOM over an 8-hour
data-taking run is shown in Fig.~\ref{fig:rapcal_cable_len}, with a
standard deviation of 0.6 ns.


\begin{figure}[!h]
 \centering
 \includegraphics[width=0.6\textwidth]{graphics/dom/rapcal/tcal_hist_22-60.pdf}
 \caption{Distribution of one-way cable delays from multiple RAPCal
   measurements on DOM 22-60 (bottom of string 22).}
 \label{fig:rapcal_cable_len}
\end{figure}

The time calibration through the entire data aquisition and software
processing chain is verified using the LED flashers. During
commissioning, all 12~LEDs on each DOM are flashed simultaneously at
maximum brightness and the arrival timea of the first photons at the DOM above the flashing DOM
are recorded. Given the vertical spacing of 17~m on a standard IceCube
string and the group index of
refraction of  1.356 in ice at 400~nm, the expected light travel time
from the flashing DOM to the DOM above is 77~ns. In DeepCore, the DOM
vertical spacings are 10~m and 7~m, corresponding to light travel
times of 45~ns and 32~ns respectively. The mean light travel
time to the DOM above for all flashing DOMs in ice is shown in
Figure~\ref{fig:flashertiming}. The mean arrival time agrees with the
expectation for the DeepCore DOMs. For the standard DOMs, the oberved
light travel time is about 3~ns longer than the expected light travel
time, due to the effects of scattering in the ice over the longer distance. Muons are also used to
verify the time calibration, including the absolute time difference
between the IceTop surface stations and the in ice DOMs~\cite{IC3:perf}.

\begin{figure}[!h]
  \captionsetup[subfigure]{labelformat=empty}
  \centering
  \includegraphics[width=0.8\textwidth]{graphics/dom/rapcal/flashmean.pdf}
  \caption{Time from flasher to DOM above for 17~m vertical spacing
    (red), 10~m vertical spacing (blue) and 7~m vertical spacing
    (green). The expected light travel time in ice for each distance is marked with
    vertical lines.}
  \label{fig:flashertiming}
\end{figure}

\subsection{\label{sec:domeff} DOM Optical Efficiency}

A baseline value for the photon detection efficiency was established
by combining PMT measurements at 337~nm and a separate model of
wavelength- and angle-dependent effects.  Absolute sensitivity
measurements were performed on 13 IceCube PMTs, using photons that
were Rayleigh scattered by 90$^{\circ}$  from a primary 337~nm laser
beam of known intensity~\cite{ICECUBE:PMT}. The results agreed well
with independent Hamamatsu measurements of sensitivity in the range
270~nm - 730~nm, which were then normalized to the 33~7nm measurement.  The resulting quantum efficiency at the center of the PMT and at 390~nm is 25\%.  A full simulation model for the DOM includes the wavelength dependence of the PMT response, optical absorption in the DOM glass and gel, discriminator threshold effects and photocathode nonuniformity.   The angle dependence is dominated by the shape of the photocathode and its response variation away from the center, which was measured for 135 PMTs~\cite{ICECUBE:PMT}.   

The laboratory-based efficiency model was supplemented with \textit{in situ} measurements using Cherenkov light from muons in the ice.  In one study, low energy muons (median energy 82~GeV) with well understood light emission were selected to illuminate the downward-facing PMTs in the ice as directly as possible. The numbers of photons detected at different distances from these muons were then compared to predictions of the simulation model~\cite{IC3:ereco}.  Based on this and other \textit{in situ}  analysis, the central value for the efficiency was adjusted upward by 10\% in the simulation, compared to the baseline.  For physics analysis, DOM efficiency is generally included as a nuisance parameter with prior uncertainty of 10\%,  which includes other uncertainties related to generation and propagation of the light. Additional laboratory measurements on assembled DOMs are in progress, including wavelength and angle dependent effects, and are expected to reduce uncertainties~\cite{ICECUBE:DOMEFF}.

The absolute calibration at 33~7nm was performed at room temperature
on a small subset of IceCube PMTs. The relative efficiency of all
assembled DOMs was separately measured as part of production testing,
using a 405~nm pulsed laser and a system of fibers and diffusers to
illuminate DOMs evenly within 50$^{\circ}$ of the optical axis.  Using
this system, the relative efficiency of DeepCore DOMs (high quantum
efficiency type) was measured to be higher by a factor 1.39 compared
to standard IceCube DOMs, agreeing well with a specified value of
1.40. Further \textit{in situ}  studies using muons yielded a factor of
1.35~\cite{ICECUBE:DC}, which is an effective value including the
Cherenkov spectrum folded with the different wavelength sensitivity
curves of the two types of PMTs.  The production testing system also
established the efficiency change from room temperature to
-40$^{\circ}$~C is less than 1\% when gain is maintained at the design value of $10^7$.  

\subsection{Baseline Stability}

The beacon hits from which the digitizer baselines are derived are
monitored continuously throughout the year. The average value of the
beacon baselines are very stable, with shifts of no more than
0.2~counts year to year, which corresponds to 0.018~mV in the FADC and
0.025~mV in the high gain ATWD channel. The single photoelectron
charge peak is stable to within 0.05~PE. The baseline shifts from May
2015 to April 2016 are shown in
Figure~\ref{fig:baseline_stability_2015}. 

\begin{figure}[!h]
 \centering
 \includegraphics[width=0.6\textwidth]{graphics/dom/reliability/baseline_stability_2015.pdf}
 \caption{Distribution of shifts in baseline values in all ATWD
   channels and the FADC for all DOMs between May 2015 and April
   2016. The DOM configuration was unchanged during this period.}
 \label{fig:baseline_stability_2015}
\end{figure}

Every year when the detector
is recalibrated, adjustments in the SPE discriminator threshold can
cause shifts of up to 0.6~counts (0.54~mV) in the FADC
baselines. ATWD baselines are unaffected by the SPE discriminator
setting. Correcting recorded waveforms for the effect of transformer
coupling as described in Sec.~\ref{sec:waveformcal} has
the side effect that small DC offset errors are converted into
apparent PMT current that increases linearly in time over the course
of a few microseconds. The effect is stronger for DOMs with old
type toroid transformers, where a baseline error of 0.6~counts can
turn an actual deposited charge of 1~PE into a measured charge of over
2~PE with an unphysical time structure. The observed distortion in a
simulated single photoelectron charge due to FADC baseline shifts is
shown in Figure~\ref{fig:charge_fadcshift}  Therefore, whenever the
discriminator thresholds are changed, the FADC baselines are measured
again and the values used for calibration are refreshed. As long as
the discriminator thresholds are unchanged, the baselines are stable
to within 0.2~counts
as shown above, and no charge distortion is seen at that level of
baseline stability.

\begin{figure}[!h]
 \centering
 \includegraphics[width=0.8\textwidth]{graphics/dom/reliability/charge_fadcshift.pdf}
 \caption{Reconstructed charge from a simulated single photoelectron
   deposit as a function of FADC baseline shift, for both old and new
   toroid DOMs. The shaded region indicates the observed range of
   baseline variation from the nominal value in ice; no observable
   distortion in the charge spectrum is seen at these values.}
 \label{fig:charge_fadcshift}
\end{figure}

The baselines are sensitive to radio frequency interference (RFI). In
2009, RF signals from a radar transmitter broadcasting at 46.3~MHz
appeared as sinusoidal or spiky signatures in the waveform
baselines. Also in 2009, a DOM 68-42 ``Krabba'' was damaged by a too-high
voltage setting during DOM calibration, and appeared to begin
sparking, causing sinusoidal waveforms to appear in the baselines of
neighboring DOMs.
%more figures go here: waveforms from Krabba disaster and Meteor radar

\subsection{Gain Stability}

The gain stability of the DOM, or the stability of the amplified charge
from converted photons, depends on a number of factors including stability
of the PMT high voltage, Main Board channel amplifiers, and the digitizers.
We can examine these subsystems using both historical calibration results
as well as by tracking the SPE charge during data-taking.

The electronic gain stability of the DOM includes variations in the
front-end electronic amplifiers and the digitizers themselves.  The
stability is checked by comparing the Main Board channel amplifier gains
from sets of calibrations taken from 2011 to 2016 (see
Fig.~\ref{fig:domcal_ch_gain}).  From year to year, the amplifier gain
calibration is repeatable to 0.1\%, 0.2\%, and 0.5\% in the high-gain,
medium-gain, and low-gain channels respectively.  Since detector completion
in 2011, a small systematic shift of $-0.3\%$ is visible in the low-gain
channel, but this is corrected by updating the calibration constants of
each DOM.

% JK: DOMCal History.ipynb
\begin{figure}[!h]
  \captionsetup[subfigure]{labelformat=empty}
  \centering
  \subfloat[]{\includegraphics[width=0.5\textwidth]{graphics/dom/reliability/channel_gain_shift_2016_2015.pdf}}
  \subfloat[]{\includegraphics[width=0.5\textwidth]{graphics/dom/reliability/channel_gain_shift_2016_2011.pdf}}
  \caption{Fractional DOM channel amplifier gain shifts, from 2015 to
    2016 (left) and 2011 to 2016 (right).  Channels 0, 1, and 2 are
    high-gain, medium-gain, and low-gain respectively.}
  \label{fig:domcal_ch_gain}
\end{figure}

The PMT gain stability is monitored during data-taking using the
single photoelectron peak of the charge distribution on each DOM due
to cosmic ray muons. A
Gaussian + exponential function is fit to the peak region as in Figure~\ref{fig:spe_fit_thresh} and the mean of the Gaussian is
tracked throughout the year. The threshold, defined as the point which
contains 1\% of the charge in the histogram, is also tracked through
the year. The peak position is calibrated to 1~PE
and is stable to within 0.01~PE for 95\% of all DOMs as shown in
Figure~\ref{fig:gain_spe_stability}. Over 99\% of DOMs show no
measurable change in the threshold as long as the discriminator
thresholds are unchanged; these settings are only changed once per year.

\begin{figure}[!h]
 \centering
 \includegraphics[width=0.6\textwidth]{graphics/dom/reliability/chargedist.pdf}
 \caption{Charge distribution on a typical in-ice DOM. The threshold
   is marked in red and a gaussian + exponential fit to
   the SPE region is shown in blue. The mean of the gaussian is used
   to monitor the gain stability.}
 \label{fig:spe_fit_thresh}
\end{figure}

\begin{figure}[!h]
  \captionsetup[subfigure]{labelformat=empty}
  \centering
  \subfloat[]{\includegraphics[width=0.5\textwidth]{graphics/dom/reliability/mean_value_2015.pdf}}
  \subfloat[]{\includegraphics[width=0.5\textwidth]{graphics/dom/reliability/meandiff_2015.pdf}}
  \caption{Distribution of the mean of the Gaussian fit to the SPE
    peak (left) and the shift in this value in every in-ice DOM between  May 2015 and April
   2016.}
  \label{fig:gain_spe_stability}
\end{figure} 

There are on the
order of 12~DOMs which show unpredictable, abrupt shifts in the SPE peak
position of 0.05~PE or more. Figure~\ref{fig:gainshift_spe} shows the time history of the
SPE peak position of one of these DOMs over 4~months. The peak shift
corresponds to increases or decreases in the MPE scaler rate,
indicating that the SPE peak shift is indeed caused by a change in the DOM
gain. However, the SPE scaler rate is stable, indicating that the
probability to detect single photons is unchanged.

\begin{figure}[!h]
 \centering
 \includegraphics[width=0.8\textwidth]{graphics/dom/reliability/gainshift.pdf}
 \caption{Mean of the Gaussian fit to the SPE peak (top) and the SPE
   scaler rate (middle) and MPE
   scaler rate (bottom) from July 2015 to November 2015 for a DOM
   which shows unpredictable gain shift behavior. The DOM
   configuration was unchanged during this period.}
 \label{fig:gainshift_spe}
\end{figure}

\subsection{Optical Sensitivity Stability}

The detector response in IceCube is calibrated with low energy muons as
described in \cite{IC3:ereco}. The detector response is monitored in each run using the track
detection probability (TDP) calculated from high
multiplicity muon tracks with more than 30 hits in IceCube. The muon
tracks are reconstructed using the likelihood methods described in
\cite{IC3:ereco}; charge and time information from the DOM under study excluded
from the reconstruction. The TDP is
defined for each DOM as the ratio of the number of detected tracks
within 100~m of the DOM to the total number of tracks within 100~m of
the DOM. This ratio depends both on the optical properties of the ice
near the DOM and the optical efficiency of the DOM. We do not attempt
to separate
these effects in the TDP, but rather use the TDP to monitor the
overall stability of the detector response. Figure~\ref{fig:tdp} shows the TDP on
string 80, which includes both standard and HQE DOMs; the TDP is
20\% - 25\% higher for HQE DOMs than for neighboring standard
DOMs, whereas the quantum efficiency is 35\% higher. The TDP is stable to within 1\% since 2012, when the baselines
were stabilized by being set in the DAQ configuration. Figure~\ref{fig:tdp} shows
the difference in the TDP for all DOMs between a run in 2012 and a run
in 2015.
%Figure goes here: TDP on string 80, showing the jump in DOMs 30-43
%which are HQE
%Figure goes here: TDP difference between 2012 run (after April 29)
%and 2015 run.

\begin{figure}[!h]
  \captionsetup[subfigure]{labelformat=empty}
  \centering
  \subfloat[]{\includegraphics[width=0.5\textwidth]{graphics/dom/reliability/tdp_onestring.pdf}}
  \subfloat[]{\includegraphics[width=0.5\textwidth]{graphics/dom/reliability/tdpcomparison.pdf}}
  \caption{Left: Track detection probability in string 80, with a
    mixture of standard DOMs (red crosses) and HQE DOMs (blue
    circles). Right: Shift in track detection probability for all in
    ice DOMs between 2012 and 2015; standard DOMs are in blue and
    DeepCore DOMs are in red. The mean of the Gaussian fit to
    the TDP shift in the standard DOMs is -0.1\% and the means of the Gaussian fit to
    the TDP shift in the DeepCore DOMs is 0.05\%.}
  \label{fig:tdp}
\end{figure}

The detector response stability is also measured with the {\it in
  situ} light sources in IceCube. Both the in-ice calibration laser
\cite{IC3:SC} and the flasher LEDs show less than 1\% difference in the total
charge collected between 2012 and 2015. 

\subsection{\label{sect:darknoise}Dark Noise}

The vast majority of background hits result from dark noise, i.e. effects that lead to the emission of an electron from the cathode of the PMT in the absence of an external photon. The total hit rate of DOMs with normal quantum efficiency PMTs is on average \SI{560}{\hertz} and \SI{780}{\hertz} for high QE DOMs. 
The dark noise is composed of two major components. Electronics noise and radioactive decays in the material form the uncorrelated noise pulses of Poissonian nature with a rate between \SI{230}{\hertz} and \SI{250}{\hertz} that follows the Richardson law for thermal emission. 
The remaining component is the so-called correlated noise
with a hit rate that increases with decreasing temperature from \SI{280}{\hertz} to \SI{340}{\hertz} (Figure \ref{fig:dom_darknoise_vs_temperature}). 
This temperature dependent noise rate profile was acomplished by combining a measured temperture profile of the the South Pole ice cap \cite{price2002temperature} with a fit of the Poissonian expectation of the total dark noise rate to every individual DOM.

\begin{figure}
 \begin{minipage}[t]{0.45\linewidth}
 \centering
  \includegraphics[width=\textwidth]{graphics/dom/performance/darknoise/HitRatevsTemp_inice_nomuons_nofit_bigfont.pdf}
 \caption{Dark noise rate in IceCube as a function of temperature, obtained from hitspooling
 data. Each data point represents the average of 12 DOM layers from 78 strings (DeepCore excluded)}
 \label{fig:dom_darknoise_vs_temperature}
 \end{minipage}
\hfill
 \begin{minipage}[t]{0.45\linewidth}
 \centering
  \includegraphics[width=\textwidth]{graphics/dom/performance/darknoise/DarkNoise_Layer2Doms.pdf}
 \caption{Time interval between successive hits for all next-to-top layer DOMs (DeepCore excluded).}
 \label{fig:darknoise_deltaT}
 \end{minipage}
\end{figure}

The correlated noise component manifests itself in an overabundance of short time intervals between hits in a single DOM compared to the expectation from random (Figure \ref{fig:darknoise_deltaT}). The short time intervals are correlated and clustered in bursts with an average number of hits per burst rising from \num{3.3} at \SI{-10}{\celsius} to \num{3.8} at \SI{-30}{\celsius}. A detailed study with hitspooling data shows that the phenomenology of the correlated noise component in IceCube is in general in good agreement with results reported in the literature with a satisfying physical explanation still to be determined. Best candidate for the source of the entire process is luminescence triggered by radioactive decay of elements like Thorium and Uranium in the glass of the DOM.
The various sources of dark noise present in a DOM are best visible when histogramming the time between subsequent raw data hits ( or hitspool data, see Section \ref{sec:domhub_hitspool}) in a DOM on a logarithmic scale, as shown in Figure \ref{fig:darknoise_deltaT_components}. An overview of the various noise components is also given in Table \ref{tab:noise}. The so far unmentioned afterpulses, a common feature of PMTs, is attributed to ionization
of residual gases by electrons that were accelerated in between the dynodes. 

IceCube uses the result of these noise studies to build a parametrized noise model that is needed for an accurate detector simulation \cite{larson2013simulation}.

\begin{table}[h!]
\caption{Characteristics of noise components in IceCube DOMs, adapted from \cite{stanisha_noise_14}.}
  \centering
  \footnotesize
\begin{tabularx}{\textwidth}{lXXX}
\toprule
Noise Component& Origin & Distribution & Parameters \\
\midrule
Afterpulses & PMT & Gaussian & $\mu = \SI{6}{\micro\second}$ \newline $\sigma = \SI{2}{\micro\second}$\\
Uncorrelated & Thermal noise\newline Radioactive Decay & Poissonian & $\lambda = \sim \SI{250}{\hertz}$\\
Correlated & Luminescence (?) & Log-normal & $\mu = \num{-6} [\log_{10}(\frac{\Delta T}{s})]$ \newline $\sigma = \num{0.85} [\log_{10}(\frac{\Delta T}{s})]$\\
\bottomrule
\end{tabularx}
\label{tab:noise}
\end{table}

\begin{figure}[!h]
 \centering
  \includegraphics[width=0.8\textwidth]{graphics/dom/performance/darknoise/SingleDOM_HitSpool_Hits_deltaT_fits_example.pdf}
 \caption{Histogram of time differences between successive hits from HitSpool data of
DOM 15-27 (blue) on a logarithmic scale in order to visualize the different noise components
(without prepulses which are comparatively insignificant) \cite{heereman2015hitspooling}.}
 \label{fig:darknoise_deltaT_components}
\end{figure}


The evolution of the dark noise rate over time was investigated using data from the supernova scaler stream \cite{IC3:supernova, briedel_phd}. There is a exponential decay over the course of the data set which is caused by a decreasing triboluminescence in general arising from a `freeze-in` procedure of newly deployed DOMs. As the ice refreezes after deployment, it becomes cleaner and impurities introduced during the drill process (mainly trapped gas) are pushed inwards into a column nearly at the center of the drill hole. This mechanical stress requires breaking bonds in the ice which in turn can lead to triboluminescence. This effect is studied preferably
in the scaler stream since the supernova analysis, compared to most other analyses, relies on low energy events buried in the noise level.  
This effect is especially recognizable in the standard deviation of the scaler distribution. The standard deviation decreases by 25\% over the course of the three years, and is not noticeably effected by the seasonal modulation of the cosmic ray induced atmospheric muon rate. Changes in the mean are initially dominated by the decay component and later by the seasonal variation of atmospheric muons. 


\begin{figure}[!h]
 \centering
 \includegraphics[width=0.95\textwidth]{graphics/dom/performance/darknoise/briedel1.png}
 \caption{Mean and variance of the supernova scaler distribution for the entire detector over the course of the first three
years of the completed IceCube \cite{briedel_phd}.}
 \label{fig:noise_over_time_briedel}
\end{figure}


The above mentioned 'freeze-in' related noise rate decay is even more clearly visible when we concentrate only on string deployed in the last drill season of IceCube, see Figure \ref{fig:noise_over_time_briedel_lastseasondepoyed}. Since the dark noise components are not correlated over several DOMs but are intrinsic to each individual DOM, the dark noise decay is not prominent in hits where neighboring DOMs are hit as well in a given time range (called HLC hits). Single DOM (or SLC) hits, on the other hand, represent exactly the decreasing noise rate behavior in the standard deviation of their corresponding SLC rate, as shown in Figure \ref{fig:slc_over_time_briedel}.

\begin{figure}[!h]
 \centering
 \includegraphics[width=0.95\textwidth]{graphics/dom/performance/darknoise/briedel4.png}
 \caption{Mean and standard deviation of the scaler rate of strings deployed in the last deployment season (austral summer of 2010/2011) \cite{briedel_phd}.}
 \label{fig:noise_over_time_briedel_lastseasondepoyed}
\end{figure}


\begin{figure}[!h]
 \centering
 \includegraphics[width=0.95\textwidth]{graphics/dom/performance/darknoise/briedel2.png}
 \caption{Mean and standard deviation of the HLC rate distribution \cite{briedel_phd}.}
 \label{fig:hlc_over_time_briedel}
\end{figure}

\begin{figure}[!h]
 \centering
 \includegraphics[width=0.95\textwidth]{graphics/dom/performance/darknoise/briedel3.png}
 \caption{Mean and standard deviation of the SLC rate distribution. The decay and jump in both quantities result from a change in the DOM deadtime by 10\%. \cite{briedel_phd}.}
 \label{fig:slc_over_time_briedel}
\end{figure}

\subsection{Special Devices}

FIX ME: add some info


