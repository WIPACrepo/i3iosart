%!TEX TS-program = pdflatex
%!TEX root = i3det-top.tex
%!TEX encoding = UTF-8 Unicode

\section{The Digital Optical Module}
\label{sec:dom}
\textsl{(Chris Wendt; 10 pages)}

\subsection{A Functional Description of the DOM}

\subsection{Mainboard}
Cite the DAQ article\cite{ICECUBE:DAQ}.

\subsection{The Photomultiplier}
Cite the PMT article\cite{ICECUBE:PMT}.

\subsection{High Voltage}

\subsection{Flasher Board}

\subsection{Pressure Housing and Optical Gel}

\subsection{Mu-metal Magnetic Shield}

\subsection{Cable Penetrator}

\subsection{Mechanical Mounting - Harness Assembly}

\subsection{Production and Testing}

\subsection{Calibration}

\subsubsection{DOMCal}

\subsubsection{Flasher Calibrations}

\subsection{Performance and Reliability}
We have over $N$ DOM years in ice.  What can be said about 
the reliability?  This section could be quite important.

\subsubsection{DOM Survivability}

\subsubsection{Electronic Gain Stability}

\subsubsection{Baseline Stability and Noise}

\subsubsection{PMT Gain and Discriminator}

\subsubsection{Optical Sensitivity Stability}

\subsubsection{Dark Noise}

Figure: rate vs time at several depths, some curves showing typical strings and other curves showing worst outliers
Figure: rate vs temperature, can again include separate curve(s) for outlier string(s) (use David H’s plot)
time correlation important at low temp, explain counting with dead time (David H?)
experience with SHDR; one (?) DOM emitted flashes that could be seen on nearby DOMs, otherwise no impact from “flashing” PMTs affecting detector trigger; no bursts or dropouts of noise seen in SPE scalers, that would also be a way to see light-emitting discharge events in the same DOM where it occurred (need careful study of monitoring and/or SN data to prove that point); refer back to gain stability re smaller changes in noise rate
hitspooling studies (David H) if not elsewhere
